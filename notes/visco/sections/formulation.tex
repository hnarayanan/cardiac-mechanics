\section{The formulation}
\label{formulation}

From \citep{Holzapfel:2009bb}, we have the following relationship for
the than the Cauchy stress,

\begin{equation}
  \begin{split}
    J\Bsigma &=   2 \psi_{1} \bB
                + 2 \psi_{2} \left(I_{1} \bB - \bB^{2}\right)
                + 2 \psi_{3} I_{3} \bone
                + 2 \psi_{4f} \bff\otimes\bff
                + 2 \psi_{4s} \bs\otimes\bs\\
        &\quad  + \psi_{8fs} \left( \bff\otimes\bs + \bs\otimes\bff \right)
                + \psi_{8fn} \left( \bff\otimes\bn + \bn\otimes\bff \right),
  \end{split}
  \label{cauchy-stress}
\end{equation}

\noindent where $\psi_{\iota} = \frac{\partial \psi}{\partial
  I_{\iota}}$. Since we are interested in working in the reference
configuration, we first need to recast this constitutive relationship
in terms of the second Piola-Kirchhoff stress tensor, $\bS(\bC)$. For
this, we use the relation $\bS = \bF^{-1} J \Bsigma \bF^{-\mathrm{T}}$
in conjunction with the following transformation rules:

\begin{equation*}
  \begin{split}
    \bF^{-1} \bone \bF^{-\mathrm{T}} &= \bC^{-1}\\
    \bF^{-1} \bB \bF^{-\mathrm{T}} &= \bone\\
    \bF^{-1} \bB^{2} \bF^{-\mathrm{T}} &= \bC\\
    \bF^{-1} \left(\ba\otimes\bb\right) \bF^{-\mathrm{T}} &= \ba_{0} \otimes \bb_{0},
  \end{split}
\end{equation*}

\noindent where $\ba$ and $\bb$ can be either $\bff$, $\bs$ or
$\bn$. Applying these rules to Eq.~\ref{cauchy-stress} results in the
following definition for the second Piola-Kirchhoff stress,

\begin{equation}
  \begin{split}
    \bS &=   2 (\psi_{1} + \psi_{2} I_{1}) \bone
           - 2 \psi_{2} \bC
                + 2 \psi_{3} I_{3} \bC^{-1}
                + 2 \psi_{4f} \bff_{0}\otimes\bff_{0}
                + 2 \psi_{4s} \bs_{0}\otimes\bs_{0}\\
        &\quad  + \psi_{8fs} \left( \bff_{0}\otimes\bs_{0} + \bs_{0}\otimes\bff_{0} \right)
                + \psi_{8fn} \left( \bff_{0}\otimes\bn_{0} + \bn_{0}\otimes\bff_{0} \right),
  \end{split}
  \label{second-piola-kirchhoff-stress}
\end{equation}

\noindent in terms of the (pseudo) invariants of the right
Cauchy-Green tensor, $\bC$:

\begin{equation}
  \begin{split}
    I_{1} &= \mathrm{tr}(\bC)\\
    I_{2} &= \frac{1}{2}\left[I_{1}^{2} - \mathrm{tr}\left(\bC^{2}\right)\right]\\
    I_{3} &= \mathrm{det}(\bC)\\
    I_{4a} &= \ba_{0}\cdot(\bC\ba_{0})\\
    I_{8ab} &= \ba_{0}\cdot(\bC\bb_{0}).
  \end{split}
\end{equation}

Looking ahead, we introduce the tensor $\overline{\bC} =
J^{-\frac{2}{3}} \bC$ with corresponding (pseudo) invariants,
$\overline{I}_{\iota}$, and partial derivatives of strain energy,
$\overline{\psi}_{\iota} = \frac{\partial \psi}{\partial
  \overline{I}_{\iota}}$. In terms of these quantities, let us define
the {\em fictitious} second Piola-Kirchhoff stress motivated by
Eq.~\ref{second-piola-kirchhoff-stress}:

\begin{equation}
  \begin{split}
    \overline{\bS} &= 2 \left(\overline{\psi}_{1} + \overline{\psi}_{2} \overline{I}_{1} \right) \bone
                    - 2 \overline{\psi}_{2} \overline{\bC}
                    + 2 \overline{\psi}_{4f} \bff_{0}\otimes\bff_{0}
                    + 2 \overline{\psi}_{4s} \bs_{0}\otimes\bs_{0}\\
            &\quad  + \overline{\psi}_{8fs} \left( \bff_{0}\otimes\bs_{0} + \bs_{0}\otimes\bff_{0} \right)
                    + \overline{\psi}_{8fn} \left( \bff_{0}\otimes\bn_{0} + \bn_{0}\otimes\bff_{0} \right).
  \end{split}
\end{equation}

\noindent We use this fictitious stress to construct a decomposition
of the total stress into volumetric and isochoric parts.

\begin{equation}
  \begin{split}
    \bS &= \bS_{\mathrm{vol}} + \bS_{\mathrm{iso}}\\
    \bS_{\mathrm{vol}} &= J \frac{\partial \psi_{\mathrm{vol}}(J)}{\partial J} \bC^{-1}\\
    \bS_{\mathrm{iso}} &= J^{-\frac{2}{3}} \mathbb{P}:\overline{\bS},\; \mathrm{where}\; \mathbb{P} = \mathbb{I} - \frac{1}{3} \bC^{-1}\otimes\bC
  \end{split}
\end{equation}



%                        + 2 \overline{\psi}_{3} I_{3} \bC^{-1}


The arguments in \citep{Holzapfel:2009bb} lead us to the following
strain energy function:

\begin{equation}
  \psi = \frac{a}{b}\exp[b(I_{1} - 3)] + \frac{a_f}{2 b_f}
\end{equation}



%

% Local Variables:
% TeX-master: "visco"
% mode: latex
% mode: flyspell
% End:
