\section{The formulation}
\label{formulation}

\subsection{Converting from the Cauchy stress to the Second Piola Kirchhoff stress}
\label{reformulating-stresses}

From \citep{Holzapfel:2009bb}, we have the following relationship for
the than the Cauchy stress,

\begin{equation}
  \begin{split}
    J\Bsigma &=   2 \psi_{1} \bB
                + 2 \psi_{2} \left(I_{1} \bB - \bB^{2}\right)
                + 2 \psi_{3} I_{3} \bone
                + 2 \psi_{4f} \bff\otimes\bff
                + 2 \psi_{4s} \bs\otimes\bs\\
        &\quad  + \psi_{8fs} \left( \bff\otimes\bs + \bs\otimes\bff \right)
                + \psi_{8fn} \left( \bff\otimes\bn + \bn\otimes\bff \right),
  \end{split}
  \label{cauchy-stress}
\end{equation}

\noindent where $\psi_{\iota} = \frac{\partial \psi}{\partial
  I_{\iota}}$. Since we are interested in working in the reference
configuration, we first need to recast this constitutive relationship
in terms of the second Piola-Kirchhoff stress tensor, $\bS(\bC)$. For
this, we use the relation $\bS = \bF^{-1} J \Bsigma \bF^{-\mathrm{T}}$
in conjunction with the following transformation rules:

\begin{equation*}
  \begin{split}
    \bF^{-1} \bone \bF^{-\mathrm{T}} &= \bC^{-1}\\
    \bF^{-1} \bB \bF^{-\mathrm{T}} &= \bone\\
    \bF^{-1} \bB^{2} \bF^{-\mathrm{T}} &= \bC\\
    \bF^{-1} \left(\ba\otimes\bb\right) \bF^{-\mathrm{T}} &= \ba_{0} \otimes \bb_{0},
  \end{split}
\end{equation*}

\noindent where $\ba$ and $\bb$ can be either $\bff$, $\bs$ or
$\bn$. Applying these rules to Eq.~\ref{cauchy-stress} results in the
following definition for the second Piola-Kirchhoff stress,

\begin{equation}
  \begin{split}
    \bS &=   2 (\psi_{1} + \psi_{2} I_{1}) \bone
           - 2 \psi_{2} \bC
                + 2 \psi_{3} I_{3} \bC^{-1}
                + 2 \psi_{4f} \bff_{0}\otimes\bff_{0}
                + 2 \psi_{4s} \bs_{0}\otimes\bs_{0}\\
        &\quad  + \psi_{8fs} \left( \bff_{0}\otimes\bs_{0} + \bs_{0}\otimes\bff_{0} \right)
                + \psi_{8fn} \left( \bff_{0}\otimes\bn_{0} + \bn_{0}\otimes\bff_{0} \right),
  \end{split}
  \label{second-piola-kirchhoff-stress}
\end{equation}

\noindent in terms of the (pseudo) invariants of the right
Cauchy-Green tensor, $\bC$:

\begin{equation}
  \begin{split}
    I_{1} &= \mathrm{tr}(\bC)\\
    I_{2} &= \frac{1}{2}\left[I_{1}^{2} - \mathrm{tr}\left(\bC^{2}\right)\right]\\
    I_{3} &= \mathrm{det}(\bC)\\
    I_{4a} &= \ba_{0}\cdot(\bC\ba_{0})\\
    I_{8ab} &= \ba_{0}\cdot(\bC\bb_{0}).
  \end{split}
\end{equation}

\subsection{General finite strain viscoelasticity theory}
\label{general-viscoelasticity-theory}

We decompose the deformation into isochoric and volumetric parts by
introducing the  tensor $\overline{\bC} = J^{-\frac{2}{3}} \bC$, with
corresponding (pseudo) invariants, $\overline{I}_{\iota}$. We further
introduce a set of internal state variables, $\BGamma_{\alpha},\;
(\alpha = 1, \ldots, m)$. In terms of these quantities, we decompose
the total strain energy into volumetric and isochoric parts, with the
isochoric part having a time-independent (elastic) and time-dependent
component (viscoelastic).

\begin{equation}
  \psi = \psi_{\mathrm{vol}}^{\infty}(J) + \psi_{\mathrm{iso}}^{\infty}(\overline{\bC})
       + \sum_{\alpha = 1}^{m} \gamma_{\alpha}(\overline{\bC}, \BGamma_{\alpha})
\end{equation}

\noindent We use this split to correspondingly construct a
decomposition of the total stress into volumetric and isochoric parts.

\begin{equation}
  \begin{split}
    \bS &= \bS_{\mathrm{vol}}^{\infty} + \bS_{\mathrm{iso}}^{\infty} + \sum_{\alpha = 1}^{m} \bQ_{\alpha}\\
    \bS_{\mathrm{vol}}^{\infty} &= J \frac{\partial \psi_{\mathrm{vol}}^{\infty}(J)}{\partial J} \bC^{-1}\\
    \bS_{\mathrm{iso}}^{\infty} &= J^{-\frac{2}{3}} \mathrm{Dev}(\overline{\bS}) = J^{-\frac{2}{3}}
    \left[\overline{\bS} - 1/3 \left(\overline{\bS}:\bC\right)\bC^{-1}\right],
  \end{split}
  \label{stress-decomposition}
\end{equation}

\noindent In Eq.~\ref{stress-decomposition}, $\overline{\bS}$ is the
{\em fictitious} second Piola-Kirchhoff stress, defined in terms of
partial derivatives of isochoric part of the strain energy with
respect to the modified invariants, $\overline{\psi}_{\iota} =
\frac{\partial \psi_{\mathrm{iso}}^{\infty}}{\partial
  \overline{I}_{\iota}}$. Motivated by
Eq.~\ref{second-piola-kirchhoff-stress}, it has the following form:

\begin{equation}
  \begin{split}
    \overline{\bS} &= 2 \left(\overline{\psi}_{1} + \overline{\psi}_{2} \overline{I}_{1} \right) \bone
                    - 2 \overline{\psi}_{2} \overline{\bC}
                    + 2 \overline{\psi}_{4f} \bff_{0}\otimes\bff_{0}
                    + 2 \overline{\psi}_{4s} \bs_{0}\otimes\bs_{0}\\
            &\quad  + \overline{\psi}_{8fs} \left( \bff_{0}\otimes\bs_{0} + \bs_{0}\otimes\bff_{0} \right)
                    + \overline{\psi}_{8fn} \left( \bff_{0}\otimes\bn_{0} + \bn_{0}\otimes\bff_{0} \right).
  \end{split}
\end{equation}

\noindent And each of the internal stress-like variables,
$\bQ_{\alpha}\; (\alpha = 1, \ldots, m)$, are governed by an ordinary
differential equation of the form:

\begin{equation}
  \dot{\bQ_{\alpha}} + \frac{\bQ_{\alpha}}{\tau_{\alpha}} =
  \beta_{\alpha} \dot{\bS}_{\mathrm{iso}}^{\infty}(\overline{\bC}),
  \quad \bQ_{\alpha}|_{t=0} = \bzero,
\end{equation}

\noindent where $\tau_{\alpha}$ and $\beta_{\alpha}$ are parameters
relating to the viscoelasticity of the material.

\subsection{Specific forms of the strain energy function suitable for the myocardium}
\label{specific-forms}

The arguments in \citep{Holzapfel:2009bb} lead us to the following
strain energy function:

\begin{equation}
  \psi = \frac{a}{b}\exp[b(I_{1} - 3)]
       + \sum_{\iota=f, s} \frac{a_{\iota}}{2 b_{\iota}}\{\exp\left[b_{\iota}\left(I_{4\iota} - 1\right)^{2}\right] - 1\}
       + \frac{a_{fs}}{2 b_{fs}} \left[\exp\left(b_{fs}I_{8fs}^{2}\right) - 1\right]
\end{equation}

\noindent We use this to define the isochoric part of our strain
energy function,

\begin{equation}
  \psi_{\mathrm{iso}}^{\infty} = \frac{a}{b}\exp[b(\overline{I}_{1} - 3)]
       + \sum_{\iota=f, s} \frac{a_{\iota}}{2 b_{\iota}}\{\exp\left[b_{\iota}\left(\overline{I}_{4\iota} - 1\right)^{2}\right] - 1\}
       + \frac{a_{fs}}{2 b_{fs}} \left[\exp\left(b_{fs}\overline{I}_{8fs}^{2}\right) - 1\right].
\end{equation}

\noindent For the volumetric part, we use a form commonly used when
modelling rubber-like solids \citep{Holzapfel:1996}:

\begin{equation}
  \psi_{\mathrm{vol}}^{\infty} = \kappa \left[\frac{1}{\beta^{2}}\left(\beta\ln J + \frac{1}{J^{\beta}} - 1\right)\right]
\end{equation}



%

% Local Variables:
% TeX-master: "visco"
% mode: latex
% mode: flyspell
% End:
