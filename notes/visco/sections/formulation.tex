\section{The formulation}
\label{formulation}

From \citep{Holzapfel:2009bb}, we have:

\begin{equation}
  \begin{split}
    J\Bsigma &=   2 \psi_{1} \bB
                + 2 \psi_{2} \left(I_{1} \bB - \bB^{2}\right)
                + 2 \psi_{3} I_{3} \bone
                + 2 \psi_{4f} \bff\otimes\bff
                + 2 \psi_{4s} \bs\otimes\bs\\
        &\quad  + \psi_{8fs} \left( \bff\otimes\bs + \bs\otimes\bff \right)
                + \psi_{8fn} \left( \bff\otimes\bn + \bn\otimes\bff \right)
  \end{split}
  \label{cauchy-stress}
\end{equation}

\noindent Since we are more interested in the second Piola-Kirchhoff
stress tensor, $\bS(\bC)$, than the Cauchy stress, $\Bsigma(\bB)$, we
first need to recast this constitutive relation. For this, we use the
relation $\bS = \bF^{-1} J \Bsigma \bF^{-\mathrm{T}}$ in conjunction with the following transformation
rules:

\begin{equation*}
  \begin{split}
    \bF^{-1} \bone \bF^{-\mathrm{T}} &= \bC^{-1}\\
    \bF^{-1} \bB \bF^{-\mathrm{T}} &= \bone\\
    \bF^{-1} \bB^{2} \bF^{-\mathrm{T}} &= \bC\\
    \bF^{-1} \ba\otimes\bb \bF^{-\mathrm{T}} &= \ba_{0} \otimes \bb_{0},
  \end{split}
\end{equation*}

\noindent where $\ba$ and $\bb$ can be either $\bff$, $\bs$ or
$\bn$. Applying these rules to Eq.~\ref{cauchy-stress} results in the
following definition for the second Piola-Kirchhoff stress:

\begin{equation}
  \begin{split}
    \bS &=   2 \psi_{1} \bone
           + 2 \psi_{2} \left(I_{1} \bone - \bC\right)
                + 2 \psi_{3} I_{3} \bC^{-1}
                + 2 \psi_{4f} \bff_{0}\otimes\bff_{0}
                + 2 \psi_{4s} \bs_{0}\otimes\bs_{0}\\
        &\quad  + \psi_{8fs} \left( \bff_{0}\otimes\bs_{0} + \bs_{0}\otimes\bff_{0} \right)
                + \psi_{8fn} \left( \bff_{0}\otimes\bn_{0} + \bn_{0}\otimes\bff_{0} \right)
  \end{split}
\end{equation}

%

% Local Variables:
% TeX-master: "visco"
% mode: latex
% mode: flyspell
% End:
